%!TEX root = ../main.tex

\section{Preprocessing}

The preprocessing step regards the transforming of questions in an \texttt{XML}
file to processed questions in a \texttt{.csv} file. The question \texttt{.csv}
file will consist of a list of word indices from the title and body including
a list of tag indices for each question.
This transforming process also includes disregarding questions that does not have
any of the top $N$ tags attached.
At the same time two other \texttt{.csv} files are created: One containing all
unique words in the extracted questions, and one containing the unique tags used.

The processing of each questions contains the following steps (code can be found
in \cref{app:preprocess-text})
\begin{enumerate}
  \item Replace all links with \textit{<link>}
  \item Remove certain unwanted symbols
  \item Remove suffix from words (e.g. \textit{haven't} $\rightarrow$ \textit{have})
  \item Remove line breaks
  \item Replace digits with \textit{<digit>}
  \item Remove double whitespaces
  \item Reduce words to their word stem (e.g. \textit{lines} $\rightarrow$ \textit{line})
  \item Lemmatize words (e.g. \textit{better} $\rightarrow$ \textit{good})
\end{enumerate}

Finally the unique words used as word dictionary were filtered by removing
words occuring in more than $50\%$ of the questions and words occuring in less
than $0.1\%$ of the questions. Also english stop words were removed making use
of the \texttt{NLTK} library.

All these steps are used in order to reduce the dimensionality of the word space
without removing much information. Here the assumption is, that e.g. words
like \textit{better} and \textit{good} kind of adds the same meaning to the
sentence, and the same with e.g. two numbers.
